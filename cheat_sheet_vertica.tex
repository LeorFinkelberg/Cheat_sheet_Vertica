\documentclass[%
	11pt,
	a4paper,
	utf8,
	%twocolumn
		]{article}	

\usepackage{style_packages/podvoyskiy_article_extended}


\begin{document}
\title{Сборник заметок\\по аналитической колоночной СУБД с массово-параллельной архитектурой без разделения ресурсов Vertica}

\author{}

\date{}
\maketitle

\thispagestyle{fancy}

\tableofcontents

\section{Установка СУБД Vertica}
Скачать Community Edition версию \texttt{Vertica} можно по ссылке \url{https://www.vertica.com/download/vertica/community-edition/}. В разделе Community Edition xx.x.x Virtual Machine следует клинуть на Open Virtualization Format. После чего начнется скачивание OVA-файла Vertica.

OVA-файл это пакет, содержащий файлы, используемые для описания виртуальной машины, который включает файл дескриптора \texttt{.OVF}, необзятельный файл манифеста \texttt{.MF}, файлы сертификатов и другие связанные файлы.

OVA-файлы это файлы Open Virtual Appliance, которые еще иногда называют файлами Open Virtual Application или файлами Open Virtualization Format Archive. Они используются программами виртуализации для хранения различных файлов, связанных с виртуальной машиной.

Другими словами, OVA-файл -- это открытое устройство виртуализации, содержащее сжатую версию устанавливаемой виртуальной машины.

Скачав OVA-файл остается только запустить его, например, в VirtualBox. Виртуальная машина \texttt{vertica\_community\_edition-xx.x.x-x.ova} будет содержать собственно Vertica Community Edition, VMart EXample Database, VMart Managment Console, Vertica Administration Tools and vsql и Vertica CE VM User Guide. Начать работать с платформой можно по руководству \url{https://www.vertica.com/docs/VMs/Vertica_CE_VM_User_Guide.pdf}.

В целом диалект SQL \texttt{Vertica} моло чем отличается от диалекта PostgreSQL, но есть и нюансы, с которыми можно ознакомиться на странице документации \url{https://www.vertica.com/docs/10.0.x/HTML/Content/Home.htm}.

Для работы с этой СУБД через помощью Python API есть своя библиотека \texttt{vertica\_python}\footnote{Устанавливается как обычно с помощью менеджера пакетов \texttt{pip}: \texttt{pip install vertica-python}} \url{https://github.com/vertica/vertica-python}

\begin{lstlisting}[
style = ironpython,
numbers = none
]
import vertica_python

conn_info = {
	'host' : '127.0.0.1',
	'port' : 5433,
	'user' : 'some_user',
	'password' : 'some_password',
	'database' : 'a_database',
	'kerberos_service_name' : 'vertica_krb',
	'kerberos_host_name' : 'vlcuster.example.com'
}

with vertica_python.connect(**conn_info) as conn:
	# do things
\end{lstlisting}

Вариант с баллансировкой нагрузки
\begin{lstlisting}[
style = ironpython,
numbers = none
]
import vertica_python

conn_info = {
	'host' : '127.0.0.1',
	'port' : 5433,
	'user' : 'some_user',
	'password' : 'some_password',
	'database' : 'vdb',
	'connection_laod_balance' : True
}

# Server enables load balancing
with vertica_python.connect(**conn_info) as conn:
	cur = conn.cursor()
	cur.execute('SELECT NODE_NAME FROM V_MONITOR.CURRENT_SESSION')
	print('Client connects to primary node:', cur.fetchone()[0])
	cur.execute("SELECT SET_LOAD_BALANCE_POLICY('ROUNDROBIN')")

with vertica_python.connect(**conn_info) as conn:
	cur = conn.cursor()
	cur.execute('SELECT NODE_NAME FROM V_MONITOR.CURRENT_SESSION')
	print('Client redirects to node:', cur.fetchone()[0])
\end{lstlisting}

А с помощью библиотеки VerticaPy \url{https://github.com/vertica/VerticaPy} к хранящимся в СУБД данным можно применять модели машинного обучения. Например
\begin{lstlisting}[
style = ironpython,
numbers = none	
]
import vertica_python
from verticapy.learn.datasets import load_titanic
from verticapy.learn.model_selection import cross_validate
from verticapy.learn.ensemble import RandomForestClassifier

# Connection using all the DSN information
conn_info = {
	'host': "10.211.55.14", 
	'port': 5433, 
	'user': "dbadmin", 
	'password': "XxX", 
	'database': "testdb"
}

cur = vertica_python.connect(** conn_info).cursor()

vdf = load_titanic(cursor = cur)

# Data Preparation
vdf["sex"].label_encode()["boat"].fillna(method = "0ifnull")["name"].str_extract(' ([A-Za-z]+)\.').eval("family_size", expr = "parch + sibsp + 1").drop(columns = ["cabin", "body", "ticket", "home.dest"])["fare"].fill_outliers().fillna()

# Model Evaluation
cross_validate(RandomForestClassifier("rf_titanic", cur, max_leaf_nodes = 100, n_estimators = 30), 
vdf, ["age", "family_size", "sex", "pclass", "fare", "boat"], 
"survived", 
cutoff = 0.35)
\end{lstlisting}


\section{Вставка нескольких строк в таблицу}

\texttt{Vertica} не поддерживает вставку нескольких строк в одной инструкции \texttt{INSERT INTO} с помощью виртуальных таблиц \texttt{VALUES} как, например, PostgreSQL. Тем неменее существует способ вставить в таблицу несколько строк за раз. Сделать это можно с помощью ключевого слова \texttt{UNION}
\begin{lstlisting}[
style = sql,
numbers = none	
]
INSERT INTO packages(package_name, solver_type)
SELECT 'Ansys', 'direct'
UNION
SELECT 'Nastran', 'iterative'
UNION
SELECT 'Abaqus', 'direct'
\end{lstlisting}



% Источники в "Газовой промышленности" нумеруются по мере упоминания 
\begin{thebibliography}{99}\addcontentsline{toc}{section}{Список литературы}
	\bibitem{juba:2019}{ \emph{Джуба С.}, \emph{Волков А.} Изучаем PostgreSQL 10. -- М.: ДМК Пресс, 2019. -- 400 с.}
\end{thebibliography}

%\listoffigures\addcontentsline{toc}{section}{Список иллюстраций}

\lstlistoflistings\addcontentsline{toc}{section}{Список листингов}

\end{document}
